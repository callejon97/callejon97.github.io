\documentclass[letter,10pt]{article}
\usepackage{TLCresume}
\begin{document}

%====================
% EXPERIENCE A
%====================
\subsection{{ALGORITHMS ENGINEER IN THE SPACE SITUATIONAL AWARENESS (SSA) TEAM \hfill Apr. 2023  --- Currently}}
\subtext{GMV \hfill Tres Cantos, Comunidad de Madrid, Spain}
\begin{zitemize}
\item Development of astrodynamics GMV C++ library for building and maintaining a space catalogue of objects.
\begin{zitemize}
    \item Low-level implementation of features in the library, and extensive validation through testing.
    \item High-level analysis of the functionality and limitations of the algorithms implemented.
    \item International working environment with team members working from Germany, France, and Spain .
\end{zitemize}
\end{zitemize}
%=====================
% EXPERIENCE B
%====================
\subsection{{INTERN IN THE SPACE SITUATIONAL AWARENESS (SSA) TEAM \hfill Jun. 2022  --- Apr. 2023}}
\subtext{GMV \hfill Tres Cantos, Comunidad de Madrid, Spain}
\begin{zitemize}
\item Performing Master's thesis: \href{https://repository.tudelft.nl/islandora/object/uuid:f472201e-0e32-4b9e-8aa3-04521908396a}{Assimilation of Swarm C atmospheric density observations into NRLMSISE-00}.
\begin{zitemize}
    \item Literature study on atmospheric density models, and data assimilation approaches.
    \item Data assimilation approach to improve the accuracy of density models at several space weather conditions, altitudes, and with several satellite geometries, implemented in Python, and included in C++ library. Analysis of the accuracy improvement.
    \item Preliminary results presented in \href{https://conference.sdo.esoc.esa.int/proceedings/neosst2/paper/90/NEOSST2-paper90.pdf}{NEO-SST 2 conference}.
\end{zitemize}
\end{zitemize}
%====================
% EXPERIENCE C
%====================
\subsection{{INTERN IN THE ADVANCED CONCEPTS TEAM (ACT) \hfill Jul. 2020  --- Nov. 2020}}
\subtext{European Space Agency (ESA) \hfill Noordwijk, Zuid Holland, The Netherlands}
\begin{zitemize}
\item Main task: create and develop optimisation challenges in the web platform \href{https://optimize.esa.int/}{optimize}.
\item Three challenges created: Jupiter Icy Moons Explorer (JUICE) mission design, Traveling Salesman Problem (TSP) based on space debris recovery, and interferometry reconstruction.
\end{zitemize}

%====================
% EXPERIENCE E
%====================
%\subsection{{ROLE / PROJECT E \hfill MMM YYYY --- MMM YYYY}}
%\subtext{company E \hfill somewhere, state}
%\begin{zitemize}
%\item In lobortis libero consectetur eros vehicula, vel pellentesque quam fringilla.
%\item Ut malesuada purus at mi placerat dapibus.
%\item Suspendisse finibus massa eu nisi dictum, a imperdiet tellus convallis.
%\item Nam feugiat erat vestibulum lacus feugiat, efficitur gravida nunc imperdiet.
%\item Morbi porta lacus vitae augue luctus, a rhoncus est sagittis.
%\end{zitemize}

\end{document}